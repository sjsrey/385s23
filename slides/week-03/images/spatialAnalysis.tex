
\documentclass[nototal,handout]{beamer}
\mode<presentation>
{
  \usetheme{Madrid}
  \setbeamercovered{transparent}
}

\usepackage{verbatim}
\usepackage{fancyvrb}
\usepackage[english]{babel}
\usepackage[latin1]{inputenc}
\usepackage{times}
\usepackage{tikz}
\usepackage[T1]{fontenc}
\usepackage{graphicx} %sjr added
\graphicspath{{figures/}}
\usepackage{hyperref}

\author{\textsc{Sergio Rey}}
\institute[ASU]{\textbf{GPH 483/598}\\\textbf{Geographic Information Analysis}\\School of Geographical Sciences\\Arizona State University\\Fall 2010}
\title[GPH 483/598]{Introduction to Spatial Analysis}
\subtitle{}
\date[Spatial Analysis]{}

% Delete this, if you do not want the table of contents to pop up at
% the beginning of each subsection:
\AtBeginSubsection[]
{
  \begin{frame}<beamer>
    \frametitle{Outline}
    \tableofcontents[currentsection,currentsubsection]
  \end{frame}
}


% If you wish to uncover everything in a step-wise fashion, uncomment
% the following command: 
\beamerdefaultoverlayspecification{<+->}
\begin{document}
\begin{frame}
  \titlepage
\end{frame}
\begin{frame}
  \frametitle{Outline}
  \tableofcontents[pausesections]
  % You might wish to add the option [pausesections]
\end{frame}



\section{Course Introduction} 

\subsection{Objectives} 

\begin{frame}
	\frametitle{Course Objectives}
 \begin{itemize}
 \item  Introduction to fundamentals of ESDA
 \item  Conceptual background
 \item  Hands-on
 \end{itemize}
 \end{frame} 

\subsection{Content and Structure} 

\begin{frame}
	\frametitle{Components}
 
\begin{block}{Four Sections}
 \begin{itemize}
 \item  Introduction and Background
 \item  Point Patterns
 \item  Geostatistics
 \item  Spatial Autocorrelation
 \end{itemize}
 \end{block} \end{frame} 

\begin{frame}
	\frametitle{Introduction and Background}
  \begin{center}
  \begin{tabular}{|rl|}
  \hline
  Session&Topic\\
  \hline
  Jan 20&Introduction\\
  22& GIS and Spatial Analysis\\
  27&Spatial Data Taxonomy\\
  29&\emph{Lab: Introduction to GeoDa}\\
  \hline
  \end{tabular}
  \end{center}
 \end{frame} 

\begin{frame}
	\frametitle{Point Patterns}
  \begin{center}
  \begin{tabular}{|rl|}
  \hline
  Session&Topic\\
  \hline
  Feb 3&Point Pattern Analysis Basics\\
    5&\emph{Lab: Descriptive Point Pattern Analysis}\\
 10&Point Pattern Processes\\
 12&\emph{Lab: Point Pattern Analysis in \emph{R}}\\
 17&Clustering and Clusters\\
 19&Lab: Scan Statistics in StatScan\\
 24&Second Order Analysis and Point Pattern Process Modeling \\
 26&\emph{Lab: Second Order Analysis and Point Processes in R}   \\
  \hline
  \end{tabular}
  \end{center}
 \end{frame} 

\begin{frame}
	\frametitle{Geostatistics}
  \begin{center}
  \begin{tabular}{|rl|}
  \hline
  Session&Topic\\
  \hline
  Mar 3&Geostatistics Basics\\
  5&\emph{Lab:Variography in ArcGIS Geostatistical Analyst and R}\\
  10&Spring Break\\
  12&Spring Break\\
  17&Kriging\\
  19&\emph{Lab: Kriging Predition in R and Geostatistical Analyst}\\
  24&AAG\\
  26&AAG\\
  \hline
  \end{tabular}
  \end{center}
 \end{frame} 

\begin{frame}
	\frametitle{Spatial Autocorrelation}
  \begin{center}
  \begin{tabular}{|rl|}
  \hline
  Session&Topic\\
  \hline
  Mar 31&Spatial Autocorrelation Basics\\
  Apr 2&Spatial Weights\\
  7&Advanced  Weights\\
  9&\emph{Lab: Spatial Weights}\\
  14&Global Autocorrelation\\
  16&\emph{Lab: Global Autocorrelation}\\
  21&Local Autocorrelation\\
  23&\emph{Lab: Local Autocorrelation}\\
  28&Exploratory Space-Time Analysis \\
  30&\emph{Lab: STARS}\\
  \hline
  \end{tabular}
  \end{center}
 \end{frame} 

\begin{frame}
	\frametitle{Presentations}
  \begin{center}
  \begin{tabular}{|rl|}
  \hline
  Session&Topic\\
  \hline
  May 5&Presentations\\
  12&Presentations (12:10-2:00)\\
  \hline
  \end{tabular}
  \end{center}
 \end{frame} 


\section{Logistics} 

\subsection{Grading} 

\begin{frame}
	\frametitle{Grading}
 
\begin{block}{Components}
 \begin{itemize}
 \item  50\% Graded Assignments
 \item  50\% Final Project
 \end{itemize}
 \end{block} 
\begin{block}{Assignments}
 \begin{itemize}
 \item  3: Point Patterns, Geostatistics, Spatial Autocorrelation
 \item  Highest two grades used
 \item  Can resubmit (up to 2 weeks after original submission)
 \end{itemize}
 \end{block} \end{frame} 

\begin{frame}
	\frametitle{Grading 483 vs. 598}
 
\begin{block}{Undergraduate}
  For your project you can either:
 \begin{itemize}
 \item  select your own data
 \item  use data I give you
 \end{itemize}
 \end{block} 
\begin{block}{Graduate}
 \begin{itemize}
 \item  you must select your own data for your project
 \item  present assignment or supplementary reading
 \item  50=40 (assignments) + 10 (presentation)
 \end{itemize}
 \end{block} 
\begin{block}{}
  \alert{All students will present their final projects}
 \end{block} \end{frame} 

\begin{frame}
	\frametitle{Prerequisites}
  All participants are expected to have  working knowledge of spatial
  analysis concepts and to be familiar with multivariate statistics. No
  extensive GIS background beyond ArcGIS basics is needed.
 \end{frame} 

\begin{frame}
	\frametitle{Course Organization}
  The course will meet in the GIS Laboratory in 316 Schwada for both
  lectures and labs. The class time will be complemented with a
  virtual classroom supported by the \emph{moodle} software. This is a
  continued experiment but I hope you will appreciate the added
  opportunity for virtual office hours and easy access to materials.
  As this is an evolving project, any feedback on the design and
  features of the site is welcome. The course site will be announced
  in class.
 \end{frame} 

\subsection{Reading} 

\begin{frame}
	\frametitle{Readings}
 There is no required textbook for the course. Supplementary readings will be
 taken from journals and the following two textbooks:
 \begin{itemize}
 \item O'Sullivan, D.O. and D.J. Unwin (2003) \emph{Geographic Information
   Analysis}. John Wiley: New York. 
 \item de Smith, M.J., M.F. Goodchild and P.A. Longley (2008) \emph{Geospatial
   Analysis}. Available at \url{http://www.spatialanalysisonline.com/}.
 \end{itemize}
 Reading lists for each topic will be given out in class and made available on
 the moodle class web site.
 \end{frame} 

\subsection{Software} 

\begin{frame}
	\frametitle{Software}
  \centering
 \begin{tabular}{l}
  GeoDa\\
    \url{http://geodacenter.asu.edu/software/downloads}
    \\ \hline
   CrimeStat \\
    \url{http://www.icpsr.umich.edu/CRIMESTAT/download.html}
    \\ \hline
   SaTScan\\
   \url{http://www.satscan.org/}
   \\ \hline
   R\\
    \url{http://cran.r-project.org/}
    \\
    \hline
   STARS
   \\
    \url{http://regionalanalysislab.org/index.php/Main/STARS}	
    \\
    \hline
 \end{tabular}
 \end{frame} 


\section{GIS and Spatial Analysis} 

\subsection{Big Picture} 

\begin{frame}
	\frametitle{GIS Then}
 \begin{center}
 \includegraphics[width=.85\linewidth]{snowmap1.pdf}
  \end{center}
 \end{frame} 

\begin{frame}
	\frametitle{GIS Then}
 \begin{center}
 \includegraphics[width=.85\linewidth]{snowmap3.png}
  \end{center}
 \end{frame} 

\begin{frame}
	\frametitle{GIS Now}
 \begin{center}
 \includegraphics[width=.85\linewidth]{crimemap.png}
  \end{center}
 \end{frame} 

\begin{frame}
	\frametitle{GIS Functions}
 
\begin{block}{Anselin-Getis (1992) Taxonomy}
 \begin{itemize}
 \item  Input
 \item  Storage
 \item  \alert{Analysis}
 \item  Output
 \end{itemize}
  Many other taxonomies
 \end{block} \end{frame} 

\begin{frame}
	\frametitle{GIScience}
 
\begin{block}{Goodchild (1992)}
 \begin{itemize}
 \item  cross-disciplinary
 \item  \alert{central} role for spatial analysis
 \item  scientific \alert{glue}
 \end{itemize}
 \end{block} \end{frame} 

\subsection{What is Spatial Analysis?} 

\begin{frame}
	\frametitle{What is Spatial Analysis?}
 
\begin{block}{From Data to Information}
 \begin{itemize}
 \item  \alert{Beyond} mapping
 \item  \alert{added value}
 \item  transformations, manipulations and application of analytical methods to spatial (geographic data)
 \end{itemize}
 \end{block} \end{frame} 

\begin{frame}
	\frametitle{Locational Invariance}
 
\begin{block}{How Insights  Change with location}
 \begin{itemize}
 \item  spatial analysis is \alert{not} locationally invariant
 \item  the results change when the locations of the study objects change
 \item  \alert{where} matters
 \end{itemize}
 \end{block} \end{frame} 

\begin{frame}
	\frametitle{State Income Distributions 1929}
 \begin{center}
 \includegraphics[width=.65\linewidth]{income29.png}
  \end{center}
 \end{frame} 

\begin{frame}
	\frametitle{State Income Distributions 1929}
 \begin{center}
 \includegraphics[width=.65\linewidth]{density29.png}
  \end{center}
 \end{frame} 

\begin{frame}
	\frametitle{Randomized Income Distribution 1929}
 \begin{center}
 \includegraphics[width=.65\linewidth]{income29random.png}
  \end{center}
 \end{frame} 

\begin{frame}
	\frametitle{Randomized Income Density 1929}
 \begin{center}
 \includegraphics[width=.65\linewidth]{density29random.png}
  \end{center}
 \end{frame} 

\begin{frame}
	\frametitle{Locational Invariance}
 \begin{figure}[ht]
  \begin{minipage}[b]{0.4\linewidth}
  \centering
  \includegraphics[scale=0.20]{income29.png}
  \end{minipage}
  \begin{minipage}[b]{0.4\linewidth}
  \centering
  \includegraphics[scale=0.20]{density29.png}
  \end{minipage}
 \begin{minipage}[b]{0.4\linewidth}
  \centering
  \includegraphics[scale=0.20]{income29random.png}
  \end{minipage}
 \begin{minipage}[b]{0.4\linewidth}
  \centering
  \includegraphics[scale=0.20]{density29random.png}
  \end{minipage}
  \end{figure}
 \end{frame} 

\begin{frame}
	\frametitle{Spatial Autocorrelation Income 1929}
 \begin{center}
 \includegraphics[width=.65\linewidth]{moran29.png}
  \end{center}
 \end{frame} 

\begin{frame}
	\frametitle{Spatial Autocorrelation Randomized Income 1929}
 \begin{center}
 \includegraphics[width=.65\linewidth]{moran29random.png}
  \end{center}
 \end{frame} 

\begin{frame}
	\frametitle{Locational Invariance}
 \begin{figure}[ht]
  \begin{minipage}[b]{0.4\linewidth}
  \centering
  \includegraphics[scale=0.20]{income29.png}
  \end{minipage}
  \begin{minipage}[b]{0.4\linewidth}
  \centering
  \includegraphics[scale=0.20]{moran29.png}
  \end{minipage}
 \begin{minipage}[b]{0.4\linewidth}
  \centering
  \includegraphics[scale=0.20]{income29random.png}
  \end{minipage}
 \begin{minipage}[b]{0.4\linewidth}
  \centering
  \includegraphics[scale=0.20]{moran29random.png}
  \end{minipage}
  \end{figure}
 \end{frame} 

\begin{frame}
	\frametitle{Components of Spatial Analysis}
 
\begin{block}{Mapping and Geovisualization}
  \alert{showing} interesting patterns
 \end{block} 
\begin{block}{Exploratory Spatial Data Analysis}
  \alert{discovering} interesting patterns
 \end{block} 
\begin{block}{Spatial Modeling}
  \alert{explaining} interesting patterns
 \end{block} \end{frame} 

\begin{frame}
	\frametitle{Summary: Spatial Analysis}
 
\begin{block}{Beyond Mapping}
  Central role for \alert{analysis}
 \end{block} 
\begin{block}{Distinguished by Locational Variance}
  \alert{Location} matters
 \end{block} 
\begin{block}{Components}
  Showing, discovering, explaining
 \end{block} \end{frame} 


\section{EDA and ESDA} 

\subsection{Exploratory Data Analysis (EDA)} 

\begin{frame}
	\frametitle{What is EDA?}
 
\begin{block}{Philosophy}
  EDA is an approach, not simply a set of techniques, but an
  attitude/philosophy about how a data analysis should be carried
  out.
 
 Postpones the usual assumptions about what kind of model the data follow
 \end{block} 
\begin{block}{Origins}
  Tukey, J. (1977) \emph{Exploratory Data Analysis}. Addison,
  Wesely
 \end{block} \end{frame} 

\begin{frame}
	\frametitle{Components}
 
\begin{block}{Set of techniques to}
 \begin{itemize}
 \item  maximize insight into a data set
 \item  uncover underlying structures
 \item  extract important variables
 \item  detect outliers and anonalies
 \item  test underlying assumptions
 \item  suggest hypotheses
 \item  develop parsimonious models
 \end{itemize}
 \end{block} \end{frame} 

\begin{frame}
	\frametitle{EDA Techniqes}
 
\begin{block}{Statistical Graphics}
 \begin{itemize}
 \item  EDA relies heavily on statistical graphics
 \item  EDA is not identical to statistical graphics
 \item  Graphics support pattern recognition and open-minded exploration
 \item  Interactive graphcs push this even further
 \end{itemize}
 \end{block} 
\begin{block}{Quantitiatve Methods}
  Although heavily graphic in orientation, there are also a number
  of numerical techniques in EDA.
 \end{block} \end{frame} 

\begin{frame}
	\frametitle{EDA Versus Confirmatory Analysis}
 
\begin{block}{Confirmatory Analysis (e.g. regression)}
  Problem $\rightarrow$ Theory $\rightarrow$ Model $\rightarrow$ Data $\rightarrow$ Conclusion
 \end{block} 
\begin{block}{Exploratory Analysis}
  Problem $\rightarrow$ Data $\rightarrow$ Analysis $\rightarrow$ Model
 \end{block} \end{frame} 

\subsection{Exploratory Spatial Data Analysis (ESDA)} 

\begin{frame}
	\frametitle{What is ESDA?}
 
\begin{block}{Definitions}
 \begin{itemize}
 \item  Type of EDA
 \item  Extended to include spatial attributes of the data
 \end{itemize}
 \end{block} 
\begin{block}{Crossfertilization}
 \begin{itemize}
 \item  Applying classic EDA to spatial data
 \item  Developing new EDA methods for spatial data
 \item  Interactions between EDA and ESDA
 \end{itemize}
 \end{block} \end{frame} 

\begin{frame}
	\frametitle{How does ESDA fit in spatial analysis?}
 
\begin{block}{Spatial Modeling?}
 \begin{itemize}
 \item  Modeling based on assumptionss
 \item  ESDA largely model free
 \item  Matter of degree (e.g., clustering)
 \end{itemize}
 \end{block} 
\begin{block}{Mapping?}
 \begin{itemize}
 \item  Maps play a critical role in ESDA
 \item  Does a map = ESDA?
 \item  No. ESDA = map, manipulation + visualization
 \end{itemize}
 \end{block} \end{frame} 

\subsection{Geovisualization} 

\begin{frame}
	\frametitle{Geovisualization}
 
\begin{block}{Beyond Mapping}
 \begin{itemize}
 \item  Combing map and scientific visualization methods
 \item  Exploit human pattern recognition capabilities
 \end{itemize}
 \end{block} 
\begin{block}{Statistical Maps}
 \begin{itemize}
 \item  innovative map devices
 \end{itemize}
 \end{block} \end{frame} 

\begin{frame}
	\frametitle{Mapping Issues}
 
\begin{block}{How to Lie with Maps}
 \begin{itemize}
 \item  Monmonnier (1996)
 \item  many design issues
 \item  projects
 \item  human perception can be tricked
 \end{itemize}
 \end{block} \end{frame} 

\begin{frame}
	\frametitle{Visual Analytics}
 
\begin{block}{The Science of Analytical Reasoning Facilitated by Interactive Visual Interfaces}
 \begin{itemize}
 \item  NVAC 2005
 \item  science of analytical reasoning
 \item  visual representation and interaction
 \item  data representation and transformations
 \item  production, presentation and dissemination
 \end{itemize}
 \end{block} \end{frame} 

\begin{frame}
	\frametitle{Visual Analysis}
 
\begin{block}{Tools}
 \begin{itemize}
 \item  synthesize inforation
 \item  derive insights
 \item  detect the expected and discover the unexpected
 \item  understandable assessments
 \item  communicate effectively
 \item  focused on policy actions
 \end{itemize}
 \end{block} \end{frame} 

\begin{frame}
	\frametitle{Visual Explanations}
 
\begin{block}{Tufte (1997)}
  Reasoning about Evidence and Design of Graphics
 \begin{itemize}
 \item  documenting sources (metadata)
 \item  appropriate comparisons
 \item  quantify and show cause and effect
 \item  multivariate nature of analytic problems
 \item  evaluate alternative explanations
 \end{itemize}
 \end{block} \end{frame} 

\begin{frame}
	\frametitle{Choropleth Map}
 
\begin{block}{Map Counterpart of Histogram}
 \begin{itemize}
 \item  values for discrete spatial uits
 \item  choro from  choros (region) NOT chloro
 \end{itemize}
 \end{block} 
\begin{block}{Discrete Approximations}
 \begin{itemize}
 \item  intervals
 \item  continuous shading
 \end{itemize}
 \end{block} \end{frame} 

\begin{frame}
	\frametitle{Map Design Issues}
 
\begin{block}{Choice of Intervals}
 \begin{itemize}
 \item  cut points: equal interval, natural breaks
 \item  statistical criteria: equal area (quantile)
 \end{itemize}
 \end{block} 
\begin{block}{Choice of Colors}
 \begin{itemize}
 \item  important for perception of pattern
 \end{itemize}
 \end{block} \end{frame} 

\begin{frame}
	\frametitle{Income Quintiles}
 \begin{center}
 \includegraphics[width=.65\linewidth]{income29.png}
  \end{center}
 \end{frame} 

\begin{frame}
	\frametitle{Outlier Map}
 
\begin{block}{Box Map}
 \begin{itemize}
 \item  Special Quartile Map
 \item  Outliers Highighlited
 \begin{itemize}
 \item  same  criteria as a box plot
 \item  outliers added as extra categories
 \item  six instead of four categories
 \end{itemize}
 \item  Both Magnitude and Location
 \end{itemize}
 \end{block} \end{frame} 

\begin{frame}
	\frametitle{Box Map}
 \begin{center}
 \includegraphics[width=.85\linewidth]{boxmapgeoda.png}
  \end{center}
 \end{frame} 

\begin{frame}
	\frametitle{Special Maps}
 \begin{itemize}
 \item  Cartogram
 \item  Conditional Maps
 \item  Map Animation
 \end{itemize}
 \end{frame} 

\begin{frame}
	\frametitle{Cartogram}
 
\begin{block}{Objectives}
 \begin{itemize}
 \item  Correct for  misleading effect of area
 \begin{itemize}
 \item  larger area units  draw attention
 \item  change layout to reflect size other than area
 \end{itemize}
 \item  Respect topology
 \end{itemize}
 \end{block} 
\begin{block}{Circular Cartogram}
 \begin{itemize}
 \item  variable maps to area/radius of circle
 \end{itemize}
 \end{block} \end{frame} 

\begin{frame}
	\frametitle{Cartogram}
 \begin{center}
 \includegraphics[width=.85\linewidth]{cartogram.png}
  \end{center}
 \end{frame} 

\begin{frame}
	\frametitle{Conditional Maps}
 
\begin{block}{Conditional Choropleth Map (Carr)}
 \begin{itemize}
 \item  Special case of conditional plots
 \item  trellis graphs	
 \end{itemize}
 \end{block} 
\begin{block}{Conditioning}
 \begin{itemize}
 \item  along two dimensions (variables)
 \item  micromap matrix
 \item  choropleth map on dependent variable
 \end{itemize}
 \end{block} \end{frame} 

\begin{frame}
	\frametitle{Conditional Choropleth: Univariate Conditioning}
 \begin{center}
 \includegraphics[width=.85\linewidth]{conditionalchoropleth.png}
  \end{center}
 \end{frame} 

\begin{frame}
	\frametitle{Conditional Choropleth: Bivariate Conditioning}
 \begin{center}
 \includegraphics[angle=270,width=.85\linewidth]{conditionalchoropleth1.png}
  \end{center}
 \end{frame} 

\begin{frame}
	\frametitle{Map Animation}
 
\begin{block}{Map Movie}
 \begin{itemize}
 \item  location highlighted in turn
 \item  from low value to high value
 \end{itemize}
 \end{block} 
\begin{block}{Looking for pattern}
 \begin{itemize}
 \item  spatial  heterogeneity
 \item  systematic movements/locations
 \end{itemize}
 \end{block} \end{frame} 

\begin{frame}
	\frametitle{Map Animation}
 Demo
 \end{frame} 

\begin{frame}
	\frametitle{Interactive Graphics}
 
\begin{block}{Interactive View Manipulation}
 \begin{itemize}
 \item  the analyst interacts with the data
 \item  dynamic graphics
 \item  no longer passive
 \end{itemize}
 \end{block} \end{frame} 

\begin{frame}
	\frametitle{Linking and Brushing}
 
\begin{block}{Linking}
 \begin{itemize}
 \item  selection in one graph is simultaneously selected in all
 \end{itemize}
    graphs
 \end{block} 
\begin{block}{Brushing}
 \begin{itemize}
 \item  changing the selection set is dynamically updated in all graphs and maps
 \end{itemize}
 \end{block} \end{frame} 

\begin{frame}
	\frametitle{Linking}
 \begin{center}
 \includegraphics[width=.85\linewidth]{linking.pdf}
  \end{center}
 \end{frame} 

\begin{frame}
	\frametitle{Brushing a  Scatter Plot}
 \begin{center}
 \includegraphics[width=.85\linewidth]{brushspgeoda.png}
  \end{center}
 \end{frame} 

\begin{frame}
	\frametitle{Brushing a  Map}
 \begin{center}
 \includegraphics[width=.85\linewidth]{brushmapgeoda.png}
  \end{center}
 \end{frame} 

\begin{frame}
	\frametitle{Multivariate EDA}
 \begin{itemize}
 \item  Scatter Plot Matrix
 \item  Parallel Coordinate Plot
 \item  3-D Scatter Plot
 \end{itemize}
 \end{frame} 

\begin{frame}
	\frametitle{Scatter Plot Matrix}
 \begin{center}
 \includegraphics[width=.85\linewidth]{spmatrix.png}
  \end{center}
 \end{frame} 

\begin{frame}
	\frametitle{Brushing a  Parallel Coordinate Plot}
 \begin{center}
 \includegraphics[width=.85\linewidth]{brushpcpgeoda.png}
  \end{center}
 \end{frame} 

\begin{frame}
	\frametitle{Brushing in 3-D}
 \begin{center}
 \includegraphics[width=.85\linewidth]{select3d.png}
  \end{center}
 \end{frame}
\end{document}
